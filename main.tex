\documentclass[a4paper,12pt,oneside]{book}

\usepackage[T1]{fontenc}
\usepackage[utf8]{inputenc}
\usepackage[italian]{babel}
\usepackage{pifont}
\usepackage{enumitem}
\usepackage[colorlinks]{hyperref}
\usepackage{graphicx} % Required for inserting images
\usepackage{geometry}

\hypersetup{
    hidelinks,
    colorlinks=true,
    linkcolor=black,      
    pdftitle={Documentazione Galleria Condivisa},
    pdfpagemode=FullScreen,
}
\geometry{a4paper, top = 3cm, bottom = 3cm, left = 3.5cm, right = 3.5cm, heightrounded, bindingoffset = 5mm}
\pagestyle{plain}


\title{\textrm{Documentazione\\Galleria Fotografica Geolocalizzata}}
\author{Gennaro Ventrone\and Luigi Solaro\and Mariano Sommella}
\date{\textsc{\large anno accademico 2022-2023}}
%\includegraphics{unina.png}




\setcounter{secnumdepth}{3}
\setcounter{tocdepth}{2}

\begin{document}
    \begin{titlepage}
        \maketitle
    \end{titlepage}
        \frontmatter
        \tableofcontents
        \mainmatter

        
    \chapter{Specifiche Progetto}
    
    \section{Introduzione Documentazione}

    Nelle seguenti pagine si riporta la documentazione relativa ad una progettazione e implementazione di un sistema di basi di dati per la gestione di collezioni fotografiche geolocalizzate condivise.
    

    \section{Elenco Specifiche}
   Vengono qui elencate le caratteristiche che l'applicativo deve garantire:
    \begin{enumerate}
        \item \emph{Informazioni sulla foto:} Possibilità di visualizzare per ogni foto l'utente che l'ha scattata, il dispositivo utilizzato e il luogo in cui è stata scattata. Il luogo può essere identificato da coordinate geografiche o da un nome mnemonico unico all'interno del sistema.
        
        \item \emph{Identificazione dei soggetti:} Ogni foto può raffigurare diversi soggetti o utenti, in entrambi i casi devono essere identificati univocamente e categorizzati.

        \item \emph{Accesso a collezioni fotografiche:} Ogni utente ha sempre la possibilità di vedere la propria personale galleria fotografica, che comprende esclusivamente le foto scattate da lui. Un utente può partecipare a collezioni condivise con altri utenti che possono contenere foto scattate da questi utenti. 
        
         \item \emph{Accesso alla foto:} Ogni utente deve avere la possibilità di accedere solo alle proprie foto o a quelle condivise con gli altri utenti. Le foto private non verranno condivise con gli altri utenti.
         
        \item \emph{Eliminazione delle foto:} Gli utenti devono avere la possibilità di eliminare le proprie foto, ma solo l'amministratore del sistema può eliminare un utente e tutte le sue foto (eccetto le fotografie che contengono come soggetto un altro utente).

        \item \emph{Operazione di ricerca:} Il sistema deve permettere la ricerca di foto in base al luogo in cui sono state scattate o al soggetto rappresentato. Inoltre, deve essere possibile visualizzare una classifica dei 3 luoghi più immortalati.
        \item \emph{Gestione dei video:} Ogni fotografia può essere visualizzata in sequenza, andando a formare quindi un video.

        \item \emph{Sicurezza:} Il sistema deve garantire la sicurezza delle informazioni, in particolare per quanto riguarda la gestione degli account degli utenti e la protezione dei dati personali. Inoltre, il sistema deve essere in grado di gestire la concorrenza tra utenti per evitare problemi di accesso simultaneo alle stesse informazioni.
    \end{enumerate}
    \par
    
    Il sistema prevede che le categorie di utenti siano così rappresentate:
    \begin{itemize}
        \item \textbf{Utente:} Può effettuare i punti dal 1 al 7 (In particolare esiste un servizio riservato al ruolo di \emph{amministratore} nel punto 5).

        \item \textbf{Soggetto:} Può essere ritratto in una fotografia ed essere categorizzato.

        \item \textbf{Amministratore:} Può effettuare il punto 5 e avere accesso agli stessi servizi di \emph{Utente}.
    \end{itemize} 
    \chapter{Modello Concettuale}
    \section{Passaggio dal minimondo al modello concettuale}\par
    Il \emph{mini-world} è una rappresentazione semplificata del mondo reale che ci fa intuire le informazioni che devono essere salvate nel database.
    Superata la prima fase che consiste quindi nella raccolta e analisi dei requisiti, la seconda fase consiste nel tradurre in entità e relazioni mediante un costrutto grafico.\par
    In seguito alla sezione precedente si è costruito il seguente schema\par concettuale espresso mediante il Diagramma UML:
    \section{Ristrutturazione del modello concettuale}
      La terza fase della progettazione di un sistema di basi di dati consiste nel passaggio dal modello concettuale al modello logico. La ristrutturazione del modello concettuale è un processo che si occupa di migliorare il modello concettuale, in questo caso quello precedentemente realizzato, per renderlo più efficiente e aggiornato alle esigenze degli utenti.Può essere realizzato nel seguento modo:
    \subsection{Analisi delle ridondanze}
  
\subsection{Eliminazione delle Gerarchie}

    \subsection{Eliminazione di Attributi Multivalori e Composti}
    \subsection{Partizionamento/Accorpamento delle Entità}
    \subsection{Analisi degli Identificativi}
    \section{Diagramma Ristrutturato}
    \subsection{Diagramma ER}
    \subsection{Diagramma UML}
    \section{Dizionario delle Entità}
    \section{Dizionario delle Associazioni}
    \section{Dizionario dei Vincoli}

    \chapter{Progettazione Locica}
    \section{Passaggio dal Diagramma ER al Modello Logico}
    \section{Mapping delle Entità e delle Associazioni}
    \section{Schema Logico}
    \chapter{Codice SQL}
    \section{Definizioni preliminari}
    \section{Creazioni Tabelle}
    \section{View}
    \section{Trigger}
\end{document}